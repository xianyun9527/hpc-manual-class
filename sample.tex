\documentclass[cs4size, a4paper]{hpcmanual}

\title{SSH命令入门}
\author{上海交通大学网络信息中心高性能计算部}
\date{2013年7月6日}

\begin{document}
\maketitle

\tableofcontents

\section{使用SSH连接集群}

\subsection{客户端设置}

在\texttt{\textasciitilde{}/.ssh/config}添加如下内容:

\begin{verbatim}
Host github.com
HostName 127.0.0.1
\end{verbatim}

运行命令进行连接测试:

\begin{verbatim}
$ ssh -T git@github.com
\end{verbatim}

\section{常用Linux命令}

\subsection{ls:查看目录内容}

\begin{itemize}
\itemsep1pt\parskip0pt\parsep0pt
\item
  ls 无参数,区别于ls -a;
\item
  ls -a 显示所有文件,包括. .. .htaccess .bash\_history~ .bash\_profile
  .bashrc .cshrc等文件;
\item
  ls -A 跟 -a 参数的区别是~ 不显示 . ..目录;
\item
  ls -alh 这个指令意思就是 -a显示全部 -l详细列表 -h~
  (human)的意思,给人看的格式human;
\end{itemize}

\subsection{mv:移动文件}

\begin{itemize}
\itemsep1pt\parskip0pt\parsep0pt
\item
  mv hzlzh.txt ..~ 将当前目录的hzlzh.txt移动到上一级目录
;
\item
  mv test.txt hzlzh.txt 将test.txt重命名为hzlzh.txt
\end{itemize}


\end{document}
